\documentclass[12pt]{ctexart}
\usepackage{geometry}
\usepackage{pifont}
 \geometry{
 a4paper,
 total={170mm,257mm},
 left=20mm,
 top=20mm,
 }

\begin{document}
\huge Unit 4 \\
\begin{center}
\LARGE Text 4-1
\end{center}

\large
\par Dose it really matter if there are fewer species of snail or beetle in the world, if some unknown planet species ceases to exist or if the gene pool of a rare species is skrinking? In short, yes. Biodiversity is the basis of a healthy, balanced global ecology capable of sustain life on Earth. A diverse ecosystem is a stable ecosystem because it is complex and flexible enough to be self-regulating. Earth's air and water, for example, are kept pure through the action of a wide range of organisms. Even the humblest creatures play their part. Through decomposition, dead matter is recycled and often detoxified in the process. For instance, microorganisms in soil and water convert toxic ammonia to nitrate ions, which are then taken up and used by plants. The atomsphere and world's climate are stabilished by plants through photosynthesis, absorbing carbon dioxide and producing oxygen.

\par A wide variety of plant life reduces the chances of flooding and drought. Roots hold the soil together and absorb vast amount of water that is then evaporated into atomsphere through transpiration. Conversely, plants that are tolerant of low water levels help to prevent desertification by maintaining a micro-environment under their canopy that reduces evaporation and ru-off, and maintains fertility.

\par Plant pollination, seed dispersal and nutrient recycling in systems such as the nitrogen cycle all maintain healthy ecosystems, and all depends on high levels of biodiversity. Some of these systems are so effecient we have harnessed them to improve our personal environment. Sewage-treamnt works are one of the best example. Microbial decomposers are kept in huge quantities and given conditions to break down our waste into relatively harmless substances that can be safely discharged into rivers or the sea, or even used as fertiliser.

\par Heathy, varied ecosystems deliver many direct and indirect benefits. In agricultrue, the wild relatives of livestock and crops provide a reservoir of genetic diversity that can be drawn upon to develop improved breeds and varieties. This natural resource will become all the more important as the world comes to terms with climate change, providing genetic characteristics that will allow crops to thrive despite changes in temperature and rainfall.

\par Of course, plants are more than just food. For thousands of years the forest and savannah have been our medicine chest. And now the pharmaceuticals industry draws upon this huge natural resource to develop new drugs. Some 56 per cent of the top 150 prescribed drugs in the US are based on chemicals derived from plants, but only 1 per cent of the 25000 known species of tropical plant have been screened for potential pharmaceuticals use. Until we have a better idea of the deversity of plant life on Earth, we cannot know how many more life-saving drugs are waiting to be discoverd.\\

\noindent 1. The first sentence of this passage intends to tell us that \_\_\_\_?\\
A. it doesn't matter at all if there are fewer species in the world\\
B. it really matters if some species extinct unexpectedly\\
C. it obviously matters if biodiversity is destroyed\\
D. it matters little if a rare species is disappearcing\\

\noindent 2. It can be concluded from this passage that \_\_\_\_.\\
A. we can hardly survive without healthy ecosystems\\
B. biological variety is our life-support system\\
C. the gene pool of nay rare species is indispensable to human beings\\
D. any species in nature should be protected from being destroyed\\

\noindent 3.The example of sewage-treatment works is used to illustrate that \_\_\_\_.\\
A. healthy ecosystems may help us improve environment\\
B. even sewage can be changed into relatively harmless substance\\
C. sewage cannot be used as if fertilizer without being properly treated\\
D. it is not safe to discharge sewage into rivers or the sea before it is treated\\

\noindent 4.This passage is mainly about \_\_\_\_.\\
A. healthy, varied ecosystem\\
B. plant diversity and agriculture\\
C. potential pharmaceutical uses of plants\\
D. the benefits resulting from biodiversity\\

\noindent 5. It can be infered from this passage that \_\_\_\_.\\
A. the future develpment of medicine is beyond imagination\\
B. the development of new drugs is dependent exclusively on plants\\
C. what we known about plant diversity is still very poor\\
D. plants are rarely considered to be the natural resource for developing medicine\\

\begin{center}
\LARGE Text 4-2
\end{center}

\large
\par Spend too long watching old movies this holiday season, and your nightlife might seem a lot less colourful. When we are surrounded by black and white imagery, we think our dreams are monochrome, says US philosopher.

\par In survey from 1950s--the golden age of black and white--most said that their dreams were never or rarely in colour, found Eric Schwitzgebel of the University of California, Riverside. Before and since, most have reported colourful dreams
\par The finding shows us how little we known our own senses, say Schwitzgebel. ``This is one piece of a general picture--our knownledge of our stream of experience is very poor."

\par Amecican dreaming in the Eisenhower era was no different to that in any other period, Schwitzgebel thinks. People were just more likely to believe that they dreamed in black and white, because that reflected the artificial dreams around them. Before the tweentieth century, dreams were often compared to paintings or tapestries.

\par We known little about what a dreaming brain is up to, comments neuroscientist Daniel Glaser of University College London. Brain scans of sleepers might show whether the brain regions that process colour vision are active during vivid dreams, he says.

\par Another possibility is that dream colours are indeterminate, in the same way that a novelist can describe something without naming its colour. They would only become coloured, or not, in our waking reconstruction of them.

\par When people say that they dream in black and white, they probably mean that they haven't noticed any colours, says psychologist Mark Blagrove of the University of Wales at Swansea, UK.

\par Black and white dreaming is a concept borrowed from technology, he agrees. ``The idea that things in dreams are in shades of grey has no meaning."

\par Our waking perceptions of colour are just as fluid. Only the central patch of the retina can see in colour, yet we perceive the whole world as coloured. Our eyes jump around, and the brain fills in the gaps with memory or guesswork. ``Our feeling that we see in colour could be akin to our perception of dreaming in colour," Glaser says.

\par The media probably influence our dreaming lives as much now as they did in the 1950s. Few people mention touch in dreams, Schwitzgebel points out that's why people pinch themselves to see if they're awake.

\par But as entertainment becomes more immersive--with virtual reality providing tactile, as well as visual, stimulation--our dreams may come to seem more touchy-feely. ``We might start thinking our dreams are really great," he says.\\

\noindent 1.Schwitzgebel believes that \_\_\_\_.\\
A. out knownledge is very pool\\
B. dreams in 1950s were never in colour\\
C. we have little knownledge about our own senses\\
D.most people tend to dream in black and white\\

\noindent 2.In the future, \_\_\_\_.\\
A. none will dream in black and white\\
B. we might not be able to see whether we are dreaming by pinching ourselves\\
C. the media will influence our dreaming much less than they do now\\
D. dreaming colours will become determinate\\

\noindent 3. Accordingto neuroscientist Daniel Glaser \_\_\_\_.\\
A. people never dream in black and white\\
B. it is meaningless to say dreaming in black and white
C. much work has to be done before we know why a brain dreams\\
D. we are incapable of perceiving the world around our colour\\

\noindent 4.It can be inferred from this passage that our dreams \_\_\_\_.\\
A. are somewhat influenced by waht we watch in waking hours\\
B. black and white dreaming results from watching old movies\\
C. people who dream in black and white are all colour-blind\\
D. what we dream are certainly what we have seen\\

\noindent 5. The best title for this passage should be \_\_\_\_.\\
A. Artificial Dreams\\
B. Our future Dreams\\
C. American Dreaming in the Eisenhower Era\\
D. Black and White Imagnary Makes Dreams Momochrome\\

\begin{center}
\LARGE Text 4-3
\end{center}

\large
\par On a more mundane level, third-generation mobile telephones, despite all the delays and the billions squndered on 3G licences by telecoms firms, are still expected to offer consumers highspeed, always on mobile internet access, complete with video, in the next few years. Rapidly proliferating ``wi-fi" networks already offer wireless access on a local basis. Tiny tracking chips called radio-frequency identification devices are being used as pet passports. Soon they will be small, powerful and cheap enough to be implanted into everything from humans to milk cartons, recording and transmitting real-time medical data or serving as a form of inventory control. Sensors of every kind, including video cameras, should also become much smaller and cheaper. Forrester Research, a technology consultancy, predicts that 14 billion such devices will be connected to the internet by 2010.

\par How rapidly such new technology is introduced will depend on a number of factors-the state of the economy, the supply of investment capital and the appetite of cousumers for new products or services. Forunes will be made and lost many times over. But whatever happens, the power of computing and communications look set to grow, and its price to fall, at a steady rate for the next few decades. That will make it possible, at least in rich countries, to record most human interactions, wherever and whenever they take place, and to store and analyse this ocean of data at low cost.

\par For the sake of argument, this survey will assume that we are heading towards a networked society of ubiquitous, mobile communications capable of constant monitoring. Whether this arrives in 20, 30, or 40 years does not really matter. The point is that destination seems not merely possible, but probable, so it is too soon to ask:what do we want this technology to do?

\par This internet has already thrown up a host of legal and political conundrms, but these are only a small foretaste of the dilemmas--about privacy, security, intellectual property and the nature of government itself--that will have to be faced over the coming decades. Tje debate jas already begun. This survey will outline some of main issues, and speculate on t he way  they are likely to go.

\noindent 1.Radio-frequency identification devices \_\_\_\_.\\
A. are being used to offer wireless access\\
B. are usally used to offer high-speed, mobile internet access\\
C. are already cheap enough to be used as passports\\
D. are expected to be used as a means to control inventory\\

\noindent 2. The first paragraph is mainly about \_\_\_\_.\\
A. the prospect of third-generation telephones\\
B. the electronic devices which can be connected to the internet\\
C. bright future of new technology application\\
D. the prediction given by Forrester Research, a technology consultancy\\

\noindent 3.The author is most concerned with the possibility that after a few decades, \_\_\_\_.\\
A. the supply of investment capital is likely to decrease considerably\\
B. consumers' appetite for new products or services will lessen tremendously\\
C. fortunes will be made and lost many times over\\
D. most human interactions can be easily monitored\\

\noindent 4.According to this passage, \_\_\_\_ may not be what people are trying to pull off.\\
A. a networked society capable of constant monitoring\\
B. the widespread application of internet and related technologies\\
C. the invention of more devices which can be connnected to the internet\\
D. mobile communications capable of internet access\\

\noindent 5.The dilemmas created by the use of internet may be about all of the following aspects, except \_\_\_\_.\\

\noindent$\parbox[l]{0.5\paperwidth}{A. copyrights\\C. constant monitoring}$$\parbox[l]{0.5\paperwidth}{B. inventory control\\D. intellectual property}$

\begin{center}
\LARGE Text 4-4
\end{center}

\large
\par In recent years, nonhuman animals have been at the centre of an intense philosophical debate. In particular, many authors have criticised traditional morality, maintaining that the way in which we treat members of other species is ethically indefensible. We routinely use animals as means to our ends--in fact, we treat them in ways in which we would deem it profoundly immoral to treat human beings. Though they are`` moral patients"--that is, beings whose treamtment may be subject to moral evaluation--their status is infinitely inferior to ours. Are such double standards warranted? And, if so, on what grounds?

\par While not being completely overlooked by philosophers, the first justification offered is powerful and widespread at the societal level, mainly due to its simplicity. To the question of what divides us from the other animals, the answer is: the fact that they are not human. On such a view, what makes the difference is the possession, or lack, of a genotype characteristic of the species Homo sapiens. Is this a good reply? No. Those appealing to species membership work within the framework of the human egalitarian paradigm. And it is just the line of reasoning that supports human equality that implies, by denying the moral relevance of race or sex membership, the rejection of the idea that species membership in itself can mark a difference in moral status. If one claims that biological characteristics like race and sex cannot play a role in ethics, how can one attribute a role to another biological characteristics such as species membership? Moral views that, while rejecting racism and sexism, accept ``speciesism"--the view that grants members of our own species special moral status--are internally inconsistent.

\par Sheer speciesism is hardly plausible. But there are more sophisticated ways of defending our current double standards to which the theoretical defenders of the status quo tend to turn. For most philosophers, it is not species membership but rather the possession of ratinality that plays a central role. We can set aside for the sake of argument the questionable assumption that rationality is a human prerogative in order to focus on the moral significance attached to ratinality.

\par Though many other defences of the doctrine of human superiority have been put forward, the appeal to species membership, the appeal to the possession of rationality as a precondition of morals, and the appeal to this very same characteristic as a means to intersubjective agreement are certainly the most basic, around which all the others revolve. If none of them can justfy maintaining nonhuman animals in their present inferior moral condition, it seems plausible to infer that our current attitude is deeply flawed.

\noindent 1.According to tranditional morality, \_\_\_\_.\\
A. animals are rarely thought of as ``moral patients"\\
B. animals should not be used as means to our ends\\
C. the way in which we treat animals is obvioursly improper\\
D. the way in which we treat animals now is undisputed\\

\noindent 2.In this passage, the author \_\_\_\_ the double standards we use to treat other species.\\

\noindent$\parbox[l]{0.5\paperwidth}{A. chanllenges\\C. justifies}$$\parbox[l]{0.5\paperwidth}{B. defends\\D. verifies}$\\

\noindent 3.The first justification offered for the double standards we use to treat other species is \_\_\_\_.\\

\noindent$\parbox[l]{0.5\paperwidth}{A. its simplicity\\C. species membership}$$\parbox[l]{0.5\paperwidth}{B. racusim and sexism\\D. huamn equality}$\\

\noindent 4. Another way used to defend the double standards is \_\_\_\_.\\

\noindent$\parbox[l]{0.5\paperwidth}{A. sheer speciesism\\C. for the sake of argument}$$\parbox[l]{0.5\paperwidth}{B. the possession of rationality\\D. the moral significance}$\\

\noindent 5. The author's attitude toward the way in which we currently treat nonhuman animals is \_\_\_\_.\\

\noindent$\parbox[l]{0.5\paperwidth}{A. impartial\\C. crucial}$$\parbox[l]{0.5\paperwidth}{B. indifferent\\D. critical}$


\end{document}