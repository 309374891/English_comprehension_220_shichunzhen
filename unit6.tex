\documentclass[12pt]{ctexart}
\usepackage{geometry}
\usepackage{pifont}
 \geometry{
 a4paper,
 total={170mm,257mm},
 left=20mm,
 top=20mm,
 }

\begin{document}
\huge Unit 6 \\
\begin{center}
\LARGE Text 6-1
\end{center}

\large

\par What might be the next alarm bell to ring? Of course, a truck bomb would intensify national nervousness by making things that are \underline{ubiquitous}--trucks--seem ominous. And high explosives directed against, say, Hoover Dam would not only complicate life in the Southwest, it would underscore the unsettling message that even big things can be pulverized. However, it is time to think about attacks using things not solid and directed against things not as solid as skyscrapers or dams.

\par Consider cyberterrorism, assaults that can be undertaken from anywhere on the planet against anything dependent on or directed by flows of information. Call this soft terrorism. Although it can put lives in jeopardy, it can do tis silent, stealthy work without tearing flesh or pulverizing structures. It can be a weapon of mass disruption rather than mass destruction, as was explained by the President's Commission on Critical Infrastructure Protection in its 1997 report on potential cyberattacks against the ``system of systems" that is modern America.

\par ``Life is good in America," the reporter says, ``because things work. When we flip the switch, the lights come on. When we turn the tap, clean water flows." Now suppose a sudden and drastic shrinkage of life's ``taken for granted" quotient. The report note that terrorit attacks have usually been against single targets individuals. crowds, buildings. But today's networked world of complexity and interconnectedness has vast new vulnerabilities with a radius larger than that of any imaginable bomb blast. Terrorists using computers might be able to disrupt information and communications systems and, by doing so, attack banking and finacial systems, energy(electricity, oil, gas), and the systems for the physical distribution of America's economic output.

\par Hijacked aircraft and powdered anthrax--such terrorist tools are crude and scare compared with computers, which are everywhere and inexpensive. Wielded with sufficient cunning, they can spread the demoralizing helplessness that is terrorism's most important intended byproduct. Computers as weapons, even more than intercontinental balistic missiles, render irrelevant the physical geopgraphy--the two broad oceans and two peaceful neighors--that once was the basis of America's sense of safety.

\par A threat is a capability joined with a hostile intent. In early summer 1997 the U.S. military conducted a threat assessment execise, code-named Eligible Receiver, to test the vulnerabilities of ``borderless cyber geopgraphy." The results confirmed that in a software-driven world, an enemy need not invade the territory,or the air over the territory, of a country in order to control or damage that country's resources.

\par The attack tools are on sale everywhere: computers, modems, software, telehpones. The attacks can shut down services or deliver harmful instructions to systems. And a cyberattack may not be promptly discovered. The report says, ``Computer intrusions do not annouce their presence the way a bomb does."\\

\noindent 1. This passage is mainly about \_\_\_\_.\\
A. future terrorist attacks\\
B. fighting against terrorism\\
C. weapons of mass disruption\\
D. weapons of mass destruction\\

\noindent 2. The underlined word ``ubiquitous" in the first paragraph most probably means \_\_\_\_.\\

\noindent$\parbox[l]{0.5\paperwidth}{A. common\\ C. solid}$$\parbox[l]{0.5\paperwidth}{B. real\\ D. tangiable}$\\

\noindent 3. Soft terrorism is mainly directed against \_\_\_\_.\\

\noindent$\parbox[l]{0.5\paperwidth}{A. dams\\ C. unsettinling messages}$$\parbox[l]{0.5\paperwidth}{B. skyscrapers\\ D. information systems}$\\

\noindent 4. America's sense of safety may be threatened to the greatest extent when \_\_\_\_.\\

\noindent$\parbox[l]{0.5\paperwidth}{A. computers\\ C. powered anthrax}$$\parbox[l]{0.5\paperwidth}{B. hijacked aircraft\\ D. intercontinental ballistic missiles}$\\

\noindent 5. The threat-assessment execise once conducted in U.S. \_\_\_\_.\\
A. obscured its cyber geography\\
B. confirmed the potential threat of cyberattacks\\
C. rendered irrelevant its physical geogrophy\\
D. enhanced America's sense of safety condierably\\

\end{document}