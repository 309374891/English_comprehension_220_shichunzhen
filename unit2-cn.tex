\documentclass[12pt]{ctexart}
\usepackage{geometry}
\usepackage{pifont}
 \geometry{
 a4paper,
 total={170mm,257mm},
 left=20mm,
 top=20mm,
 }

\begin{document}
\huge Unit 2 \\
\begin{center}
\LARGE Text 2-1
\end{center}

\large
\par 作为婴儿,我们对过去没有认识;作为成人,我们能想起十年前的事。我们对时间的认识延伸到今天早晨喂养和上周的洗澡之外。之前科学家对这种显著的转变只有一个模糊的认识,现在他们知道的更多了:哈佛大学的Conor Liston发现长期记忆开始在孩子第九个月到17个月的期间升高,伴随着脑部记忆处理区域的结构性变化。除了解释了为什么小孩不记得上个月去迪士尼的游玩,这些结果还能帮助指导将来对习惯发展和脑部变换之间关系的研究。

\par “孩子在生命中的第一年能保留多少记忆是不清楚的。”Liston说。“我们有兴趣测试这样一种假设:在一岁末两岁初的神经发育可能会导致这种记忆明显增强。”

\par 

\noindent 词汇:

cinder

reckon

reign

ever-expanding

healthy

contend

obstacle

demise

arguably

envelop

inhospitable

consume

ultimately

sever

It's arguable that...可以论证的


\end{document}
