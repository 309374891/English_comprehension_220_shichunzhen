\documentclass[12pt]{ctexart}
\usepackage{geometry}
\usepackage{pifont}
 \geometry{
 a4paper,
 total={170mm,257mm},
 left=20mm,
 top=20mm,
 }

\begin{document}
\huge Unit 5 \\
\begin{center}
\LARGE Text 5-1
\end{center}

\large

\par Foals are on their feet not long after being born. Chicks break through their shells and within hours are pecking around for food. Snakes hatch and just slithe away. Humans, on the other hand, are completely helpless at birth and remain dependent on their parents for many years.

\par So why does it take humans so long to develop? Why is such a intelligent species helpless for so many years? The quick answer, experts believe, is that humans are the most complex living system. And the more complex the system, the longer it takes to build.

\par The more involved answer has to do with evolution. It took billions of years for life evolve from single-celled microorganisms to large warm-blooded mammals. ``The ultimate cause of this evolution, the why and the how one genetic program is selected over another is not yet clear to us," says Calvin, a professor of psychiatry and behavioral sciences at the University of Washington in Seatle. But researcher have identified some of the evolutionary factors that may have affected our developmental process, he says.

\par One of the factors dates back to around half a billion years ago, when two strategies for protecting offspring emerged. One group of species under a more-is-better strategy, where species produced thousands of eggs so the odds increased that some of them would survive to reporduce. The second group of species, which includes humans, took the opposite approach. They began having the longer the young are protected and taught the tricks of the trade, so to speak, the better chance they have of surviving to reproduce.

\par Another factor is called neoteny, a devempmental trend where a juvenile appearence is retained well past biological adulthhood(the age at which we can reproduce). Just the fact that we look young and vulnerable increase our chances of being taken care of.

\par Experts say it's likely that our development also is related to the gradual increase in human brain size over millions of years. Limited by the size of the female pelvis, we evovled in a way that allows for postnate brain growth. Because the brain plays a key role in the development of the body, this adaptation may have, in turn, forced the delay in much of our growth until we are outside the womb.

\par The higher intelligence and inquisitiveness of humans, our capacity for abstract thought and ability to plan ahead also play a huge role in our slow development. ``Humans are capable of doing things that no other species can," says Calvin, and it simply takes us a while to master the tasks that we need to survive.

\par Language is a good example of this. Learning a language and the specifics of grammar, syntax and context takes years, but humans are born with an innate drive to master it.\\

\noindent 1.The proper title for this passage should be: \_\_\_\_.\\
A. Investing in Our Youngth\\
B. A General Feature of Higher Primates\\
C. The Higher Intelligence of Human\\
D. Why do We Take So Long to Develop\\

\noindent 2.It can be infered from this passge that the author must be \_\_\_\_.\\
A. a professor of psychiatry\\
B. interested in mysteries of the universe\\
C. fully aware of the ignorance of humans\\
D. disappointed to realize the holpless state of humans\\

\noindent 3. Calvin thinks that the development process of humans is \_\_\_\_.\\
A. wholly beyond comprehension\\
B. limited by the size of the female pevis\\
C. primarly related to evolution\\
D. affected merely by two significant factors\\

\noindent 4. According to this passage, the development of humans \_\_\_\_.\\
A. the strategy of inversting in youngth plays a significat role in\\
B. more-is-better strategy plays a key role in\\
C. neoteny is a dominant factor affecting\\
D. the innate drive to master language is the chief factor affecting\\

\noindent The author's attitude towards the issue of ``human's slow development" is \_\_\_\_.\\

\noindent$\parbox[l]{0.5\paperwidth}{A. amazed\\ C. curious}$$\parbox[l]{0.5\paperwidth}{B. biased\\ D. puzzled}$\\


\begin{center}
\LARGE Text 5-2
\end{center}

\large

\par One of remarkable thing of laughter is that it occurs unconsciously. You don't decide to do it. While we can consciously inhibit it, we don't consciously produce laughter. That's why it's very hard to laugh on command or to fake laughter.

\par Laugher provides powerful, uncensored insights into our unconscious. It simply bubbles up from within us in certain situations.

\par When we laugh, we alter our facial expressions and make sounds. During exuberant laughters, the muscles of the arms, legs and trunk are invovled. Laughter also requires modification in our pattern of breathing. We also know that laughter is a message that we send to other people. We know this because we rarely laugh we are alone.

\par Laughger is social and contagous. We laugh at the sound of laughter itself.

\par The first laughter appears at 3.5 to 4 months of age, long before we're able to speak. Laughter, like crying, is a way for a preverbal infant to interact with the mother and other caregivers.

\par Contrary to folk wisdom, most laughter is not about humor; it is about relationship between people. To find out when and why people laugh, I and several undergraduate research assistants went to local malls and city sidewalks and recorded what happened just before people laughed. Over a 10-year period, we studied over 2000 case of naturally occuring laughter.

\par We found that most laughter does not follow jokes. People laugh after a variety of statements such as ``Hey, John, where ya been?" ``Here comes mary," ``How did you do on the test?" and ``Do you have a rubber band?". These are certainly are not jokes.

\par We don't decide to laugh at these moments. Our brain makes the decision for us. These curious ``ha ha ha's" are bits of social glue that bond relationships.

\par Curiously, laughter seldom interrupts the sentence structure of speech. It punctuates speech. We only laugh during pauses when we would cough or breathe.

\par When we laugh, we're often communicating playful intent. So laughter has bonding function within individuals in a group. It's often posivitive, but it can be negative too. There's a different between ``laughing with" and ``laughing at." People who laugh at others may be trying to force them to conform or casting them out of the group.

\par No one has actually counted how much people of different ages laugh, but young children probably laugh the most. At ages 5 and 6, we tend to see the most exhberant laughs. Adults laugh less than children. probably because they play less. And laughter is associated with play.\\

\noindent 1.``You don't decide to do it." in the first paragraph probably means that \_\_\_\_.\\
A. you are reluctant to laugh\\
B. you laugh without consciousness\\
C. you cannot inhibit you laughter\\
D. you are not prepared for laughing in public\\

\noindent 2. When laugh when \_\_\_\_.\\

\noindent$\parbox[l]{0.5\paperwidth}{A. we are not alon\\ C. we hear somebody laughing}$$\parbox[l]{0.5\paperwidth}{B. we hear the word of laughter\\ D. we modify our pattern of breathing}$\\

\noindent 3.According to this passage, laughter is frequently triggered simply by \_\_\_\_.\\

\noindent$\parbox[l]{0.5\paperwidth}{A. jokes\\ C. what one thinks himself}$$\parbox[l]{0.5\paperwidth}{B. folk wisdom\\ D. what other people say}$\\

\noindent 4.It can be infered from the study over 2000 case of naturally occuring laughter that \_\_\_\_.\\
A. laughter is rerely natural\\
B. most laughter is social\\
C. laughter is to enliven the speech\\
D. the ability to laugh is inherent\\

\noindent 5.This passage is mainly about \_\_\_\_.\\
A. when an why we laugh\\
B. general feature of laughter\\
C. the brain mechanisms of laughter\\
D. an evolutionary perspective of laughter\\

\begin{center}
\LARGE Text 5-3
\end{center}

\large

\par The blue haze represents X-ray emissions from hot gas between galaxies in the cluster MS1054-0321, 8 billion light-years away. What confines the gas within the cluster? Some propose that it's dark matter.

\par If gravity works the way it's supposed to, then most of the universe's mass in visible, existing as what's come to be known as ``dark matter." What's the nature of that missing mas, and what does it all mean for the fate of universe? The question lead to some of  the greatest mysteries of modern physicas.

\par Scientists haven't even figured out yet how much total mass the universe contains--a no-less-weighty question that is linked to the dark matter debate. Indeed, the nature and amount of dark matter determines whether the universe itself is fated to collapse back upon itself, expand into virtual nothingness or reach a state of quilibrium.

\par Right now, the best bet is that there isn't enough matter for gravity to overcome the Big Bang, meaning that the universe's current expansion will continue forever utils there's pratically nothing left. In fact, some scientist are puzzling over data indicating that the expansion is accelerating.

\par For a long time, cosmologists worked under the assumption that there is enough matter to bring the universe into an eventual balance. Cosmologists call this balance point the critical density, and they used a variable called ``omega" to describe the proportion of the universe's actual density to the critical density.

\par If omega equals one, the universe is in balance and all is well for most theoretical physicists. But if omega is much less than one -- as appears to be the case -- then the theoreticians have a lot of explaining to do. In fact, it may indicate that we don't fully understand how gravity works after all.

\par That's why some physicists hope there's enough undetected dark matter to fill the gap.

\par Figuring out the total mass of the universe may sound like an imponderable question--but surprisingly, Lawrence and other researchers hope to come up with some conclusive answers the next decades or so. Their strategy is to measure the uneven afterglow of the Big Bang's aftermath, known as the cosmic background radiation.

\par A satellite called the Cosmic Background Explorer has made a good start toward charting that afterglow. Future spacecraftsuch a NASA's Microwave Anisotropy Probe and the Enuropean Space Agency's Planck mission will map the early universe's signature in event greater detail. By closely comparing the density differences in the background radiation, astronomers can come up with an answer for the mass question and gain some new hints as to the nature of dark matter.

\par ``I think in 10 or 15 years we will known pretty much for sure whether the universe will expand forever, collpase back on itself or just drift," said Lawrence, who is a principal investigator for one of the Planck research teams. ``That's pretty exciting. That's a question that didn't exist 100 years ago."

\noindent 1.According to this passage, the universe \_\_\_\_.\\
A. is unlikely to collapse back upon itself\\
B. is still full of mysteries to be revealed by us\\
C. will continue to expand until nothing exists in it\\
D. will reach a state to equilibrium sooner of later\\

\noindent 2.Some cosmlologists assume that \_\_\_\_.\\
A. the universe is made of dark matter\\
B. the dark matter in the universe is missing\\
C. gravity may not work the way it's supposed to\\
D. gravity is a concept that didn't exist 100 years ago\\

\noindent 3. The variable ``omega" used by cosmologists \_\_\_\_.\\
A. tends to be 1\\
B. is usually much less than 1\\
C. refers to the balance point of the universe\\
D. refers to the actual density of the univers\\

\noindent 4. Cosmologists \_\_\_\_ tge titak nass if the universe.\\
A. have figured out\\
B. have little possibility to know\\
C. are too ignorant of the nature of dark matter to estimate\\
D. are expected to find some way to figure out\\

\noindent 5. It is pointed out in the passage that \_\_\_\_ has continued to the study of the Big Bang's aftermath.\\
A. the theory of gravity\\
B. the Cosmic Background Explorer\\
C. NASA's Microwave Anisotrophy Probe\\
D. the European Space Agency's Planck mission\\

\end{document}