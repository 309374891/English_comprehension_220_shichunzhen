\documentclass[12pt]{ctexart}
\usepackage{geometry}
\usepackage{pifont}
 \geometry{
 a4paper,
 total={170mm,257mm},
 left=20mm,
 top=20mm,
 }

\begin{document}
\huge Unit 2 \\
\begin{center}
\LARGE Text 2-1
\end{center}

\large
\par As infants, we live without a sense of the past; as adults, we can recall events from decades ago. Scientists have only a vague understanding of this remarkable transition, when our sense of time expands beyond this morning's feeding and last week's bath, but now they known a bit more: Conor Liston of Harvard University has determined that the beginnings of long-term recall arise between the ninth and the 17th month of a baby's life, coinciding with structual changes in the moemory-processing regions of the brain. Besides explaining why Junior doesn't remember last month's trip to Disney World. the results should help guide future research on the link between early behavioral development and changes in the infant brain.
\par "It wasn't clear how long children in the first year of life could retain a memory of an event," Liston says. "We were intrested in testing the hypotheis that neurological developments at the end of the first year and the begining of the second year would result in a significant enhancement in this kind of memory."

\par Liston showed a simple demostration on infants ages 9, 17, or 24 months old.
\par The test results showed a huge difference between the test children who had been 9 months old when they saw the first demostration and those who had been older. Whereas 9-month-olds don't really remember a thing after four months, 17-and 24-month- olds do," Liston says. "Something is hanppening in the brain between 9 and 17 months old that enables children to encode these memories effeciently and in such a way that they be retained and retriveved after a long period of time," Liston say. Researchers believe thet changes in certain regions of the brain's frontal lobe and the hippocampus, which are associated with memory retention and retrival, drive the rapid expansion of childhood recall. Previous studies have shown that the frontal lobes in humans begin to mature during the last quart of the four year of life.
\par Liston's work may help explain why adults can rarely  rememober anything that from before thier second birthday or so. Most people simply accept this infant amnesia as a fact of life. "But it's not clear why a 40-year-old has plenty of memories for something that happened 20 years ago, but a 20-year-old has basically no memories for something that happened when he was 2 or 3," Liston says. He suggests the same brain mechanisms that were not yet able to encode long-term memories in 9-month-olds may also play some role in adults' inhability to remember events of infancy. Research still need to look at other areas of cognition-such as what role language ability plays in memory-to really fully understand why people can't remember anything that happened before 2-3 years of age. But one thing is clear: When 1-year-old Snookums claims he dosn't remember breaking the heirloom china five months ago, he's almost surely telling the truth.


\begin{center}
\LARGE Text 2-2
\end{center}
\par Though the counterfeiting of medicines, toys and aircraft and motor parts clearly poses risks to public saftey, the problem for makers of genuine products is that a large propotion agrees with Mr.Williams that priacy is great. A MORI poll in Britian found that 40$\%$ of the questioned of those admitted they might knowing buy counterfeit goods. Though it is hard to see the harm done in buying, say, a fake Gucci handbag from a steet seller, the makers of branded products and European Commission argue that this may be helping to finance crime bangs which are also involved in more dangerous activities. It also denies governments tax revenus that they need to finance public services, and it cost jobs makers of genuine goods. The multinationals gathered at Davos claimed that Russia's government is losing 200 million a year in tax due to pirating of 22 leading branded products. The European Commission, in launching its proposed crackdown, quoted a study showing that 17000 jobs a year are being lost in the EU, as legitimate manufactures lose out to imported fakes.
\par Besides anoucing proposal for tough penalties against founterfeiting, the commission is proposing that customs officers be given more power to inspect and seize goods suspected of being fakes. It is especially worried that, since most of the faked goods sold within its borders come from east of the EU, things will get much worse next year, when new member like Poland and Baltic states join the Union and their customs services become preoccupied with guarding their eastern frontiers.
\par As bussiness leader has come to realise, turing back the rising tide of priacy will require concerted action, by industry and law-enforcement agencies, on a number of fronts. Copyright holders, such as film studios and record companies, are pressing makers of hardware to incorporate anti-copying devices in their products; some are taking internet-service providers to court to froce  them to reveal the identities of large-scale of copiers of music. Nigeria's authorities, facing with wildspread counterfeiting of medicines, are leaning on the banks that lend money to priates to set up their factories.
\par But even if the European Commission's new propsoal are agreed and adopted by all member countries, this may not be enough to deter the pirates. As a report by the commission notes gloomily: "There is a realization that those engaged in priacy and counterfeiting will not be induced by an examplary action to reduce their involvement, but only by the sheer weight of successful actions taken against them, involving substantial penalties," In other words, counterfeiting is now such a profitable bussiness that, while there is a chance of getting away with it, there will be plenty prepared to run the risk.

\noindent 1.It can be infered from the passage that \_\_\_\_\\
A. there is no harm in buying faked products\\
B. faked products don't neccessarily pose any risks to public safety\\
C. only branded products may be priated\\
D. the public is fully aware of the harm done by counterfeiting\\
\noindent 2.The European commission is extraordinarily worried that\_\_\_\_ in the future.\\
A. Poland and Baltic state will join the Union\\
B. customs officers will be given more powers\\
C. new penalties against counterfeiting may not be tough enough\\
D. pirates and counterfeiting will become more uncontrollable\\
\noindent 3.The underlined word "concerted" in this paragraph most probably means\_\_\_\_.
A. agressive\\
B. concerned\\
C. combined\\
D. rigorous\\
\noindent 4. Even the member countries of the European Commission \_\_\_\_\\
A. are reluctant to reduce their involvment in piracy and counterfeiting\\
B. are hasitant to take any action against counterfeiting\\
C. may be fully prepared to run the risk of counterfeiting\\
D. may fail to arrest the pirates\\
\noindent 5.The author's attitude toward the problem of counterfeiting is \_\_\_\_\\
A. pessimistic \\
B. optimistic\\
C. subjective \\
D. objective\\

\begin{center}
\LARGE Text 2-3
\end{center}

\par The well-known "Red list" that details which species are threatened with extinction is inaccurate, according to a new assessment. It conclude the list fails to reflect the true threat to species, by not taking full account of the threat posed by people.

\par The red list, which is compiled by the World Conservation Union(IUCN), gauges a species' risk of extinction mainly on the basis of its population size, rate of decline and geographic range.

\par But Alexander Harcourt and Sean Parks at the University of California, Davis, argue that this is not enough. They compare an endangered species to a house that has been left unlocked. The house is vulnerable to burglary, but it only becomes threatened when there is a burglar nearby.

\par In the same way, a small population of animals susceptible to extinction only becomes actively threatened when it is being poached or its habitat is destroyed. Harcourt and Parks advocate modifying the Red List criteria to include local human population density.

\par Although a large number of people nearby may not in itself be a threat, they argue that hunting, pollution and habitat destruction, for example, are still likely to increase as people encroach on wildlife. What is more, data on human density is readily available. "We have the numbers, why not use them?" says Harcourt.

\par To illustrate their point, the research reassessed 200 primate species from the 1996 Red List. They found that 17 species designated as being at relatively low risk by the Red List should now be reassigned as high priority.

\par Contrary to the expectations of many, the researchers also found that two high-profile species, the gorilla and pygmy chimpanze, or bonobo, should be downgraded to a lower level of threat.

\par But Craig Hilton-Taylor, Red List Programme Officer based in Cambridge, England, says that the IUCN has already introduced a specific classification system for threats such as human density. The system runs in parallel to the main Red List classification.

\par Besides, part of the Red List's value is that you can make comparisions with past assessments, he says, and tweaking the criteria would make this impossable. "We've been asked by everyone, please don't change the system again." says Hilton-Taylor.

\par Harcourt maintains that making explicit threats part of the criteria is not only more accurate, it may also help highlight future problems. Matt Walpole, a conservation researcher at the University of Kent at Canterbury, England, agrees: "Where population data is laking, it might be a useful way of flagging up potentially threatened species."

\noindent 1.The Red List used to determine the risk of extinction a species may run by assessing all of the following, except\_\_\_\_\\
A. human population data\\
B. the species' population size\\
C. the species' rate of decline\\
D. the geographic range of the species\\
\noindent 2. An endangered species is compared to an unlocked house in order to show that\_\_\_\_\\
A. an unlocked house is vulnerable to burglary\\
B. an unlocked house is not threatened without a burglar nearby\\
C. no other comparions are more intelligible than this one\\
D. the Red List fails to reflect the threat posed by people to rare species\\
\noindent 3. In order to indicate the level of threat posed by people to rare species, \_\_\_\_ has have been introduced.\\
A. a specific classification system\\
B. a more accurate assessment system\\
C. the Red List criteria\\
D. the Red List classification\\
\noindent 4. The level of risk indicated by the Red List to each endangered species should be \_\_\_\_.\\
A. downgraded\\
B. upgraded\\
C. reassessed\\
D. kept as it is\\
\noindent 5. The proper title for this passage should be \_\_\_\_.\\
A. Data on human density\\
B. Red Alert over Rare Species\\
C. Red List Classification\\
D. Potentially Threatened Species\\

\begin {center}
\LARGE Text 2-4
\end {center}

\par Hominids get older as the years go by this year, it seems, we've turned 7 million, or at least that's the suggestion made by the skull unearthed in the Djourable Desert in Chad. The ancient skull is butting up against theories of human' evolution because of its incredible antiquity and unexpected human features.

\par A group of geologists, sedimentologists, and anthrologists reported the find last July. "The main thing in this discovery is the age," says Michel Brunet, a paleoanthrologist at the University of Poitiers in France who led the team. The specimen itself--a nearly complete skull, two portions of the jaw, and a handful of teeth--can't be dated directly because it wasn't found in the kind of sediment that allows radioactive dating; however, Brunet was able to come up with an age by comparing the bones with those found at similar sites in Kenya and Ethiopia.

\par The fossil's humanlike face and teeth and chimp-size cranium are so different from any other known hominid that Brunet and his team have denominated it a new species: Sahelanthropus tchadensis. The French team nicknamed the fellow Toumai in Goran, the language local to the site, the word means "hope of life." The hybrid of familiar face and tiny brain means Toumai probably lived just after the time when chimps and hominids were going their seperate ways. "My colleagues once thought that chimpanzees and humans diverged around 5 million or 6 million years ago," say Brunet. "But with Toumai, it's quite clear that the divergence is older--at least 7 million years, and maybe a little bit more." The evidence that he was more hominid than ape can be found in the details. The area behind the skull's thick brow ridge( which suggests it was a male) is relatively flat." In apes, chimpes, and gorillas, you have a big depression just behind this brow ridge," says Brunet. "It's quite clear that the Toumai cranium is compeletely different." Toumai's samll canines are further proof that he was more like us than an ape.

\par The famous 3-million-year-old Lucy, a.k.a Australopithecus afarensis, had a larger cranium than Toumai but a much more chimplike face. Evolution doesn't usually reverse itself, so it's unlikely that we evolved from a chimp face to a human face to a chimp face and then back to human again. The implication of this find is that many different species of hominid have walked the earth--and so our story may be more complex than previously imagined.

\noindent 1.The skull discovered in Chad proves that \_\_\_\_\\
A. the theory of human evolation is outdated\\
B. theories of human evolution are nothing bu pesudo science\\
C. the age of hominid is much longer than we usually believe\\
D. great scientific discoveries are usually unexpected\\
\noindent 2. We learn from this passage that \_\_\_\_\\
A. only a paleoanthrologist is interested in the age of an ancient skull\\
B. the age of the skull unearthed in Chad is too old to dated\\
C. radioactive dating is rerely used to date the age of ancient skull\\
D. radioactive dating is not the only way to date the age of ancient skull\\
\noindent 3. Which of the follwing can be used to prove that Toumai was more hominid than ape?\\
A. Toumai means "hope of life"\\
B. The skull of Toumai has a humanlike face and teeth\\
C. It's clear that apes and humans diverged at least 7 million year ago.\\
D. It was once thought that apes and humans diverged around 5 million or 6 million years ago.\\
\noindent 4. According to this passage, Lucy, a.k.a Australopitheus afarensis, \_\_\_\_\\
A. could not be the oldest ancestor of humans\\
B. was the oldest ancestor of humans\\
C. wasn't a species of hominid\\
D. had been evolved from Toumai\\
\noindent 5. The best title for this passage should be \_\_\_\_\\
A. An ancient skull\\
B. From Chimp to Human\\
C. Different species of Hominid\\
D. Humans Celebrate 7 Million Birthday\\
\end{document}
