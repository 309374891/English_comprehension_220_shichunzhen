\documentclass[12pt]{ctexart}
\usepackage{geometry}
\usepackage{pifont}
 \geometry{
 a4paper,
 total={170mm,257mm},
 left=20mm,
 top=20mm,
 }

\begin{document}
\huge Unit 3 \\
\begin{center}
\LARGE Text 3-1
\end{center}

\large
\par 幼鸟和哺乳动物和成人一样睡3次。新的发现表明这使的他们的大脑能适应他们经历的多种多样的新鲜事物。

\par 三沙市加利福尼亚大学的Michael Stryker教授发现被放置在新环境下的已知小猫的脑部变化的多少和猫低波睡眠(不做梦状态)紧密相关。

\par 这种关联是一个惊喜,Stryker说到,因为梦,也称作快速眼部活动(REM)睡眠,经常被认为是脑部回想和分析他们当天的经历的方式。

\par “睡眠的好处似乎在于non-REM睡眠是很有意思的,”英国的Loughborough大学的睡眠研究者Jim Horne说到。“以前REM睡眠一直被认为对脑部发育非常重要。”

\par 在睡眠期间释放的化学物质通过刺激和抑制神经增长来重塑大脑,Stryker说到。许多激素--比如睾酮素和生长激素,都是在夜间释放的,他怀疑夜间的化学信号“可能在巩固和加强醒着时的体验方面扮演重要角色。”

\par 团队研究小猫,小猫的脑部变化很快,而且对环境变化很敏感。研究者遮住猫的一只眼睛6个小时。这导致大脑视觉皮层减弱对从被遮挡的眼睛来的图像的反应,而增强了对活跃眼睛的输入反应。

\par 猫在测试期间睡了6小时后,反应神经的数量变化是没有休息的猫的两倍。睡过觉的脑部变化甚至比额外遮眼6小时的的多。

\par 这可能是睡眠的一个基础功能吗?我们能从发育中的小猫外推到完全形成的人类吗?能从可视系统到其他精神适应吗?

\par Stryker这么认为:“我们想问,例如,学习一些知识然后睡觉,能否是相比于熬一晚上更加有效的学习方式?”他说,这些结果表明沉睡能得到A。

\par 加拿大Ontario的Trent大学的心理学家Carly Smith研究睡觉如何影响人类学习和记忆。他同意在经历了新事物后的睡眠中神经可能会改变。

\par 在他的实验中,Smith也看到了低波睡眠和人们学习重复的,精巧且精确的运动模式之间的联系“可能对运动的人或者音乐学家很重要”。

\par "如果训练人们一个任务,我们发现如果打断他们的低波睡眠,他们会忘记怎么做,"Smith说道。

\noindent 词汇:

riot 暴乱;丰富多彩

novelty 新奇

remodel 重塑,改造

hormone 激素

cortex 皮层

commitment 承诺;花费

neural 神经的

extrapolate 推断

adaptation 改编版;适应

Land of Nod 沉睡之乡

delicate 精巧的


\end{document}
