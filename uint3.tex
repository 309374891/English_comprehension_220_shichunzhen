\documentclass[12pt]{ctexart}
\usepackage{geometry}
\usepackage{pifont}
 \geometry{
 a4paper,
 total={170mm,257mm},
 left=20mm,
 top=20mm,
 }

\begin{document}
\huge Unit 3 \\
\begin{center}
\LARGE Text 3-1
\end{center}

\large
\par Baby birds and mammals sleep for up to three times as long as adult. New research suggests that this may be to allow their brains to adapt to the riot of novelty they experience.

\par Michael Stryker, of the University of California, San Francisco, found that the amount of change in the brains of young cats placed in a new environment was tightly linked to the amount of slow-wave sleep-- the non-dreaming state -- the cats got.

\par This link is a surprise, says Stryker, as dreams, also known as rapid eye-movement(REM) sleep, are often thought of as the brain's way of replaying and analysing the day's experience.

\par "The finding that the benefit of sleep seems to lie in non-REM sleep is very interesting," says Jim Horne, a sleep researcher at the University of Loughborough, UK. "REM sleep has been seen as important in the maturation of the brain."

\par Chemicals released during sleep might remodel the brain by stimulating and inhibiting neural growth, Stryker says. Many hormones, such as testosterone and growth hormone, are released at night, and he suspects that nocturnal chemical signals "may really play a crucial role in consolidating and enhancing waking experience."

\par The team studied young cats, whose brains are developing rapidly and are particularly sensitive to environment change. The researchers covered one of the cats' eyes for six hours. This causes the brain's visual cortex to weaken its response to the image from the covered eye, and up its commitment to the active eye.

\par When cats slept six hours after the test period, the amount of change in their neural response was double that of cats denied a nap. Sleeper's brain changed more even than those of cats that spent an extra six hours with one eye covered.

\par Might this be a general function of sleep? Can we extrapolate from developing cats to fully formed humans, and from the visual system to other forms of mental adaptation?

\par Stryker thinks so: "We wanted to ask whether, for example, studying for an exam and then sleeping is as effective a way of learning as pulling an all-nighter." These results, he says, suggest that the route to an A-grade passes through the Land of Nod.

\par Psychologist Carly Smith, of Trent University, Ontario, Canada, studies how sleep affects human learning and memory. He agrees that similar neural changes might occur in the sleeping human brain after new experience.

\par In his experiments, Smith too has seen a link between slow-wave sleep and people's ability to learn repetitive but delicate and precise patterns of movement -- of the type that "might be important for sports people or musicians."

\par "If we train people on a task, we find that if we interrupt their [slow-wave] sleep, they don't remember how to do it," Smith says.

\noindent 1.The experiments with young cats are used in this passage to \_\_\_\_.\\
A. reveal the secret of dreaming\\
B. show how much slow-wave sleep the cats got\\
C. illustrate that sleeping may help improve learning\\
D. demonstrate then sensitivity of young cats to the environmental change\\

\noindent 2.It's point out in this passage that the amount of \_\_\_\_ may be of great importance in the maturation of the brain.\\
A. non-REM sleep\\
B. REM sleep\\
C. total sleep\\
D. brain change\\

\noindent 3.A good night's sleep might be \_\_\_\_.\\
A. a neccessory route to an A-grage\\
B. the best way to let the brain file the day's event\\
C. the unique form of mental adaptation\\
D. the effective cure of insommia\\

\noindent 4. Carlyle Smith believes that \_\_\_\_.\\
A. REM sleep doesn't benifit the brain development at all\\
B. non-REM sleep contributes a lot to the training result of a new task\\
C. there is an obscure link between sleep and the ability to learn\\
D. neural changes occurs in the brain only after new experience\\

\noindent 5. The proper title for this passage is \_\_\_\_.\\
A. Why babes sleep longer\\
B. Sleep and Brain Development\\
C. REM vs. Non-REM Sleep\\
D. To Sleep, Purchase to learn\\


\begin{center}
\LARGE Text 3-2
\end{center}

\large

\par Critics have derided a 1998 extension of American copyrights as the "Mickey Mouse Protection Act" because it stopped the early images of the Disney company's mascot from entering the public domain. But such laws, they argue, are no jokes. Extending and strenthening copyrights, the claim, will help a handful of big corporation crush creativity in the digital age. On the contrary, say Hollywood studios and big record companies. Without stronger copyright protection, a wave of piracy will destroy their industries, depriving consumers everywhere of a broad choice of movies, music and books.\\
\par Last week America's Supreme Court weighed into what is rapidly becoming a nasty worldwide battle about the scope and enforcement of copyrights, by rejecting a challenge to the 1998 law on constitutional grounds. But even as it upheld the law, the court expressed misgivings. Blistering deissents from two justices dismissed the 20-year extension of copyright as unwarranted, and even the majority's opinion hinted that Congress's decision may have been "unwise."\\
\par The court's ambivalance is understandable. The growing quarrel over copyrights is just one of the many difficult issues thrown up by the spread of the inernet and related technologies. But of all these issues, the copyright battle is becoming one of the most urgent, and bitterly fought, because it could yet determine the future character of cyberspace itself.\\
\par Both sides have a point. Digital piracy does indeed threaten to overwhelm so-called "content" industries. As the power and reach of the internet continue to grow, the illicit trading of perfect copies may well devastate the music, movie and publishing industries. The content industries want to protect themselves with anti-copying technology, backed by stronger laws. So far, they have been at loggerheads with technology firms about how to implement such schemes. But a deal between Hollywood and Silicon Vally is likely eventually. Critics are right to fear that, when such a deal is struck, it will be in the interests of big firms, not the public.\\
\par The alternative is to return to the original purpose of copyright, something no national legislature has yet been willing to do. Copyright was originally the grant of a temporary government-surppot monopoly on copying a work, not a property right. Its sole purpose was to encourage the circulation of the ideas by giving creators and publishers a short-term incentive to disseminate their work. Over the past 50 years, as a result of heavy lobbying by content industries, copying has grown to such ludicrous proportions that it now often inhibits rather promotes the circulation of the ideas, leaving thousands of old movies, records and books languishing behind a legal barrier. Starting from scratch today, no ratinal, disinterested lawmaker would agree to copyright that extend to 70 years after an author's death, now the norm in the developed world.\\

\noindent 1. By saying "such laws ... are no jokes" in the first paragraph, critics mean that \_\_\_\_.\\
A. consumers will have little choice of intellectual products\\
B. consumers will have a broad choice of movies, music and books\\
C. Mickey Mouse Protection Act isn't justifiable\\
D. piracy will destroy most entertainment industries\\
\noindent 2. The second paragraph primarily intends to say that \_\_\_\_.\\
A. America's Supreme Court totally rejected the 1998 law\\
B. America's Supreme Court went to great lengths to safegaurd the 1998 law\\
C. America's Supreme Court has conflicting attitudes toward the copyright battle\\
D. even America's Supreme Court might have done something wrong\\
\noindent 3. The music, movie and publishing industries \_\_\_\_.\\
A. have been propersing with the development of modern technology\\
B. will be devastated sooner or later by digital piracy\\
C. are rarely influenced by digital piracy\\
D. are badly endangered by digital piracy\\
\noindent 4.Copyright was originally designed to \_\_\_\_.\\
A. protect the interests of so-called "content" industries\\
B. benefit creators and publishers\\
C. faciliate the circulation of ideas\\
D. hamper the circulation of ideas\\
\noindent 5.The author's attitude toward the copyright battle is \_\_\_\_.\\
A. critical\\
B. impartial\\
C. indifferent\\
D. enthusiastic\\

\begin{center}
\LARGE Text 3-3
\end{center}

\large

\par This is because new electronic technologies deal with the very essence of human society: commucation between people. Earlier technologies, from printing to the telegraph, have done likewise, and have wrought big changes over time. But the social changes over the coming decades are likely to be much more extensive, and to happen much faster, than any in the past, because the technologies driving them are continuing to develop at a breakneck pace. More importantly, they look as if together they will be as pervasive and ubiquitous as electricity. Whether this will be for good or ill is impossible to predict, because how they are applied will be a mater of social and political choice. Many of these choices will be difficult and divisive.
\par Billios of dollars have lost betting on the idea that internet would quickly change everything from retailing to entertianment. Internet usage has continued to grow, but most doctoms have failed, and the telecommucations industry, which raced to build the infrastructure for the cyberspace, is staggering under \$1 trillion of debt. Yet it would be wrong to conclude that this is the end of the internet revolution. Boom and bust often follow the introduction of the rapdically new technologies. In the 1870s Amecria's railroad industry boomed in much the same way as the world's telecoms industry in the late 1990s, only to collpase in a similar heap of bankruptcies, accounting scandles, stockmarket losses and enormous debts, America's economy fell into recession.
\par A few years later, a reviving economy together with advances in railway engneering triggered a new wave of investment. Railroads quickly revived, changing American bussiness forever. The same sort of thing happpened when the internal combustion engine came along. In the first few years of the 20th centrury there were thousands of people tinkering with carmaking, most of whom went bust. A decade later only a handful survived, but the car was about to become the icon of progress.
\par The reason to think that the internet revolution will not only resume but accelerate is that advances in its underlying technologies show no sign of slowing down. The power of computer chips continues to race ahead. Moore's law--according to which the power of computer chip will double every 18 months--has proved to be true since 1965, when it was first propounded by Gordon More, a co-founder of Intel, a chip maker. Intel is confident that it will be able to maintain this pace of improvement in silicon for another 15 years. Recent breakthroughs by researchers at IDM and Hewlett Packard in melecular electronics lead many experts to believe that Moore's law will continue to apply for perhaps another 50 years. Similarly dramatic advances in storage and transimission technologies are also in prospect.

\noindent 1.The underlin word "ubiquitous" in the first paragraph most probably means \_\_\_\_.\\
A. omnipresent\\
B. enormous\\
C. universal\\
D. unique\\
\noindent 2. Which of the following statements is true?\\
A. The internet is likely to change everything quickly.\\
B. The advances in commucation between people drive the society forward.\\
C. The limit to the power of computer chips will never reached.\\
D. Electronic technoloies will no doubt replace the earlier technologies.\\
\noindent 3.The America's railroad industry is mentioned in the passage to illustrate that \_\_\_\_.\\
A. internet revolution will never fail\\
B. modern technology isn't neccessarily a good thing\\
C. a rising industry is likely to collapse\\
D. railroad finally became the icon of progress\\
\noindent 4. The author believes that \_\_\_\_.\\
A. there is no progress without boom and bust\\
B. all industries have to collapse first before they succeed\\
C. the internet will continue to suffer more setbacks\\
D. the internet and related technologies really will profoundly transform society\\
\noindent 5. The author's attitude toward the internet revolution is \_\_\_\_.\\
A. depressing\\
B. puzzling\\
C. optimistic\\
D. pessimistic\\


\begin{center}
\LARGE Text 3-4
\end{center}

\par The best guess of biologists is that species are disappearing between 100 and 1000 times as fast as they were before Homo sapiens arrived. But our impact is different from the mass extinctions of the past. \underline{They} wiped out whole groups of animals, notably the dinosaurs, whereas human are picking off individual species. In the past, biodiversity recovered as species spread into new ecological niches, but humans are wiping out niches as well organisms. Wildlife will have a tough time regenerating.
\par The winners after the mass extinction that finished off the dinosaurs are about to become the losers. One in four mammal species and one in eight bird species face a high risk of extinction in the near future: the population of each species is expected to fall by at least a fifth in the next 10 years. Almost all are endangered by human activity.
\par As global climate change shifts temperatures across the planet, species may not be able follow fast enough. According to the UNEP, they will have to migrate 10 times as fast as they did after the last ice age. Many won't make it.
\par Species that do up and leave move at different rates, breaking up existing communities. At high lattitudes, entire forest types are expected to disappear, to be replaced by new ones. During this transition, carbon will be lost to the atmosphere faster than it can be replaced by new growth, accelerating climate change.
\par Biodiversity is good for humans. By destroying it, we could bring the axe down on our own heads. Rural communities in more than 60 countries get much of their meat from wild animals. Overpopulation, famine and the spread of high-powered rifles are killing off these creatures. In many areas local people are going hungury. In the Congo basin, conflict has forced people to sell wild meat, putting the squeeze on creatures such as large antelopes, gorillas and chimpanzees. This bush meat trade is growing so fast it will soon be unsustainable, warns Douglas Williamson of the UN Food and Agriculture Organization.
\par Fewer species will mean fewer medicines. Three-quarters of the top 150 prescription drugs in the US are lab versions of chemicals found in plants, fungi, bacteria and vertebrates. The WHO estimates that more than 60 per cent of the world's population relies on plants for primary healthcare. There are 3000 people plants species used in birth control alone.
\par Even if we stop killing species today, nobody reading this will see wildlife restored to tis former glory, says Anne Weil of Duke University in North Carolina. Some scientists believe that the dent already made in biodiversity will take 10 million years to repair itself.

\noindent 1. The underline word "They" in the first paragraph most probably refers to \_\_\_\_.\\ 
A. the mass extinction of the present\\
B. the mass extinction of the past\\
C. the mass extinction fo the dinosaurs\\
D. the extinctions of individual species\\
\noindent 2.According to this passage, \_\_\_\_.\\
A. before Homo sapiens arrived, species disappeared much faster than now\\
B. mammal species are facing higher risk of extinction than bird species\\
C. both mammal and bird species are seriously endangered by humans\\
D. biodiversity is likely to recover in the near future as long as we drive away at it\\
\noindent 3.Global climate change is detrimental to many species because \_\_\_\_.\\
A. they can't endure high temperatures\\
B. they usally move at different rate\\
C. they tend to migrate 10 times faster than they once did\\
D. they may be incapable of adjusting themselves to it\\
\noindent 4. That biodiversity is beneficial for humans can be illustrated by the fact that \_\_\_\_.\\
A. rural commutions in numerous contries get much of their meat from wild animals\\
B. the number of plant species used in birth control is as many as 3000\\
C. many people are going hungury because of the lack of wild animals\\
D. rare creatures are more likely to be threatened by human activity\\
\noindent 5. The autor's attitude toward the issue of biodiversity is \_\_\_\_.\\
A. pessimistic \hfill B. optimistic\\
C. critical \hfill D. indifferent\\

\end{document}